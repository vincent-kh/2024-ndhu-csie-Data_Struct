\documentclass[12pt,a4paper]{article}
\usepackage{CJKutf8}
\usepackage{amsmath}
\usepackage{enumerate}
\usepackage{color}
\usepackage{multicol}
\usepackage[ruled,vlined,titlenumbered]{algorithm2e}
\usepackage[left= 0.8in,right=0.8in,top=2cm,bottom=2cm]{geometry}
\usepackage{tikz}
\begin{document}
\begin{CJK*}{UTF8}{bsmi}
\title{Electromagnetics (II) Homework 6}
\author{411025020 莊奕賢}
\maketitle

\begin{enumerate}[1.]
    \item File Tree 
    \begin{enumerate}
        \item The root of the tree is \textbf{"/user/rt/courses/"}.
        \item Internal nodes is which nodes have at least one child.
        \item \textbf{"cs016/"} have three children,\textbf{"homeworks/"}, 
        \textbf{"programs/"} and \textbf{"grades"}.\\
        And the nodes \textbf{"homeworks/"}, \textbf{"programs/"} each have three child nodes.\\
         Therefore, the node \textbf{"cs016/"} have 9 descendant nodes.
        \item The node \textbf{"cs016/"} have 1 ancestor node, \textbf{"/user/rt/courses/"}
        \item The sibling nodes of \textbf{"homeworks/"} are \textbf{"programs/"} and \textbf{"grades"}
        \item The nodes in the subtree rooted at \textbf{"projects/"} are \textbf{"papers/"}, 
        \textbf{"demos/"}, \textbf{"buylow/"}, \textbf{"sellhigh/"}, \textbf{"market/"}.
        \item The depth of the node \textbf{"papers/"} is 3.
        \item The height of the tree is 4.
        \item The output of preorderPrint(T,T.root())\\
        /user/rt/courses/ cs016/ grades homeworks/ hw1 hw2 hw3 programs/ 
        pr1 pr2 pr3 cs252/ projects/ papers/ buylow/ sellhigh/ demos/ market/ /grades
        \item The output of postorderPrint(T,T.root())\\
        grades hw1 hw2 hw3 homeworks/ pr1 pr2 pr3 programs/ cs016/ buylow sellhigh papers/ market 
        demos/ projects/ grades cs252/ /user/rt/courses/
    \end{enumerate}
    \item Expressions Tree
    $$ (5-(1+1))\times 7 = 21$$
    \begin{center}
    \begin{tikzpicture}[level/.style={sibling distance=50mm/#1}]
        \node{$\times$}
            child
            {
            node{$-$}
                child{node{$5$}}
                child
                {
                node{$+$}
                    child{node{$1$}}
                    child{node{$1$}}
                }
            }
            child{node{7}};
        \end{tikzpicture}
    \end{center}
\end{enumerate}

\end{CJK*}
\end{document}