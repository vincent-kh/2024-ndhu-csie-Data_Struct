\documentclass[12pt,a4paper]{article}
\usepackage{CJKutf8}
\usepackage{amsmath}
\usepackage{enumerate}
\usepackage{color}
\usepackage{multicol}

\usepackage[left= 0.8in,right=0.8in,top=2cm,bottom=2cm]{geometry}
\begin{document}
\begin{CJK*}{UTF8}{bsmi}
\title{Electromagnetics (II) Homework 4}
\author{411025020 莊奕賢}
\maketitle

\begin{enumerate}[1.]
    \item Vector operation
    \begin{center}
    \begin{tabular}{c|cl}
        & Operation & Vector \\ \hline\hline
    1 & insert(0,4)   & 4     \\ \hline
    2 & insert(0,3)   & 3,4   \\ \hline
    3 & insert(0,2)   & 2,3,4 \\ \hline
    4 & insert(2,1)   & 2,3,1,4 \\ \hline
    5 & insert(1,5)   & 2,5,3,1,4 \\ \hline
    6 & insert(1,6)   & 2,6,5,3,1,4 \\ \hline
    7 & insert(3,7)   & 2,6,5,7,3,1,4 \\ \hline
    8 & insert(0,8)   & 8,2,6,5,7,3,1,4 \\ \hline
    \end{tabular}
    \end{center}

    \item $kn$ access to $n$ element\\
    對於一個有$n$個元素的陣列,存取$kn$次 \\
    少於$k$次的存取的元素數量:\\
    \newline
    Min:\\
    若所有元素都存取$k$次,則少於$k$次的存取的元素數量為0。\\
    \newline
    Max:\\
    若有1個元素存取次數超過$k+n-1$次,則可能有$n-1$個元素存取次數少於$k$次。\\
\end{enumerate}

\end{CJK*}
\end{document}
