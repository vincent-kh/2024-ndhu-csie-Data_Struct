\documentclass[12pt,a4paper]{article}
\usepackage{CJKutf8}
\usepackage{amsmath}
\usepackage{enumerate}
\usepackage{color}
\usepackage{multicol}

\usepackage[left= 0.8in,right=0.8in,top=2cm,bottom=2cm]{geometry}
\begin{document}
\begin{CJK*}{UTF8}{bsmi}
\title{Electromagnetics (II) Homework 1}
\author{411025020 莊奕賢}
\maketitle

\begin{enumerate}
    \begin{multicols}{2}
        \item Stack operation
        \begin{center}
        \begin{tabular}{c|cc}
          & Operation & Stack \\ \hline\hline
        1 & push(5)   & 5     \\ \hline
        2 & push(3)   & 5,3   \\ \hline
        3 & pop()     & 5     \\ \hline
        4 & push(2)   & 5,2   \\ \hline
        5 & push(8)   & 5,2,8 \\ \hline
        6 & pop()     & 5,2   \\ \hline
        7 & pop()     & 5     \\ \hline
        8 & push(9)   & 5,9   \\ \hline
        9 & push(1)   & 5,9,1 \\ \hline
        10 & pop()    & 5,9   \\ \hline
        11 & push(7)  & 5,9,7 \\ \hline
        12 & push(6)  & 5,9,7,6 \\ \hline
        13 & pop()    & 5,9,7 \\ \hline
        14 & pop()    & 5,9   \\ \hline
        15 & push(4)  & 5,9,4 \\ \hline
        16 & pop()    & 5,9   \\ \hline
        17 & pop()    & 5     \\ \hline
        \end{tabular}
        \end{center}

        \item Queue operation
        \begin{center}
        \begin{tabular}{c|cc}
          & Operation & Queue \\ \hline\hline
        1 & enqueue(5)   & 5     \\ \hline
        2 & enqueue(3)   & 5,3   \\ \hline
        3 & dequeue()    & 3     \\ \hline
        4 & enqueue(2)   & 3,2   \\ \hline
        5 & enqueue(8)   & 3,2,8 \\ \hline
        6 & dequeue()    & 2,8   \\ \hline
        7 & dequeue()    & 8     \\ \hline
        8 & enqueue(9)   & 8,9   \\ \hline
        9 & enqueue(1)   & 8,9,1 \\ \hline
        10 & dequeue()   & 9,1   \\ \hline
        11 & enqueue(7)  & 9,1,7 \\ \hline
        12 & enqueue(6)  & 9,1,7,6 \\ \hline
        13 & dequeue()   & 1,7,6 \\ \hline
        14 & dequeue()   & 7,6   \\ \hline
        15 & enqueue(4)  & 7,6,4 \\ \hline
        16 & dequeue()   & 6,4   \\ \hline
        17 & dequeue()   & 4     \\ \hline
        \end{tabular}
        \end{center}
    \end{multicols}
        \item Deque operation
        \begin{center}
        \begin{tabular}{c|ccc}
          & Operation & Deque & return\\ \hline\hline
        1 & insertFront(3)   & 3     &\\ \hline
        2 & insertBack(8)    & 3,8   &\\ \hline
        3 & insertBack(9)    & 3,8,9 &\\ \hline
        4 & insertFront(5)   & 5,3,8,9& \\ \hline
        5 & removeFront()    & 3,8,9 &\\ \hline
        6 & eraseBack()      & 3,8   &\\ \hline
        7 & first()          & 3,8   &3\\ \hline
        8 & insertBack(7)    & 3,8,7 &\\ \hline
        9 & removeFront()    & 8,7   &\\ \hline
        10 & last()          & 8,7   &7\\ \hline
        11 & eraseBack()     & 8     &\\ \hline
        \end{tabular}
        \end{center}
    
\end{enumerate}

\end{CJK*}
\end{document}
